\documentclass[10pt]{article}

\setlength{\oddsidemargin}{-0.25in} % Left margin of 1 in + 0 in = 1 in
\setlength{\textwidth}{7in}   % Right margin of 8.5 in - 1 in - 6.5 in = 1 in
\setlength{\topmargin}{-.75in}  % Top margin of 2 in -0.75 in = 1 in
\setlength{\textheight}{9.2in}  % Lower margin of 11 in - 9 in - 1 in = 1 in

\renewcommand{\baselinestretch}{-1}
\begin{document}

\section{Common}
\subsection{Importance of Linux at Intel - Imad Sousou, Director of OTC at Intel}
\begin{itemize}
\item Imad Sousou, Director of OTC, Intel pic.twitter.com/WFFKQmCH
\item Intel heavy invests in W3C \& APIs - for example NFC API
\item Intel OSS - http://01.org
\end{itemize}

\subsection{Forward-Looking Development: The Next Evolution in Enterprise Linux Software Development - Ralf Flaxa, VP of Engineering at SUSE}
\begin{itemize}
\item  Rafl Flaxa, VP of Enineering, Suse pic.twitter.com/33nbLd9u
\item Ralf Flaxa: Talking as a Linux community dinasour - sharing his experience in Linux 
\item  21 years ago communications: E-Mail, FTP, newsgroup
\item Rafl Flaxa: "We work for different companies but we have the same spirit - we are one big Linux family" 
\item Rebase with legacy support 
\item Ralf Flaxa: "Forward looking development is a new model for enterprise Linux" 
\end{itemize}

\subsection{Research Into Open Hardware - Catarina Mota, Founder at openMaterials
\begin{itemize}
\item Not all would be software. Catarina Mota speaking about Open Hardware on Electronics
\item "The Global Village Construction Set" - using open hardware to build a city. Interesting concept presented by Mota. 
\item OSHWA - Open Source Hardware Association
\item "Arduino is the main brain of open hardware" used by "backyard brains" to change the world @openmaterials Catarina Mota at 
\item OSHW: It's more compliated than software, it's not only distribute the 2d schema. . . 
\item OSHW: 3d Printers printing pieces for make 3d printers awesome!!  They make himselfs
\end{itemize}

\subsection{Linux: Where Are We Going - Linux Creator Linus Torvalds and Intel's Chief Linux and Open Source Technologist, Dirk Hohndel}
\begin{itemize}
\item  Somebody tries to improve the world through linux - but not I (c) Linux
\item  If company says that they want make billion of \$ - I don't care (c) Linus
\item  Most interesting things - that something what I didn't expect (c) Linus
\item "I do Linux for other people in the sense that I like the community," says Linus.  Europe
\item Linus Torvalds: "I don't know what you guys are interested in." Audience member shouts: "NVIDIA!" 
\item  In linux kernel development we like argue.
\item For the kernel the direction has always come from outside. Linus. 
\item Linus Torvalds: Im not worring about the future but mantain it open all this. Legal Situations, Patents. 
\item The patent system is just broken, says Linus @ 
\item Linus: "Occasionally I worry about hardware companies..." 
\item Linus Torvalds: I mostly worry about kernel developers conflicts and the broken patent system
\item Linux: We had huge issues initially with the embedded software companies 
\item Linus Torvalds: Daily builds and tests been making but we don't have all the hardware to test it. 
\item Linus Torvalds: The lockdown of the mobiles systems make difficult to work with this embedded systems kernels. 
\item Linus: "I'm so pleased with what has happened in the ARM community in the last two years" 
\item Many questions resolving around the needs of embedded systems and supposedly too high frequency of changes in kernel 
\item Linus: "I don't think there's a lot of innovation in OSes that's worth doing" 
\item " not a lot of innovations that need doing in the OS space" says Linus ;) 
\item Linus - "sometimes the old ways are the correct ways" re op systems at 
\item There is a high threshold for removing code from the kernel, says Linus.  Europe
\item Linus: "In general we do not remove code [ from the kernel ] unless it has major problems" 
\item android situation is better now, than a year ago (c) Linus
\item Linus Torvalds: There are some vendors making wrong forks to work 
\item The Android "situation" is no different than the Red Hat or SUSE "situation" 10 years ago, says Linus.  Europe
\item Linus Torvalds: Google people are given the code from android tree but not all are been getting 
\item Google has one hundred million devices running their code. They must be doing something right. Linus. 
\item Linus Torvalds: we are all older but there is a lot of people from the university comming and working as professional 
\item Linus: "Actually performance vs. power is often not a versus" 
\item Linus at : "We are 99\% male and 1\% female in the community. Nobody knows why".
\item  We have no plan, but everybody is happy (c) Dirk
\item You can read all the Q\&A between Linus Torvalds and the  Europe audience from today's keynote here: http://bit.ly/SB2uzj
\end{itemize}

\section{System}
\subsection{FreeIPA, Jakub Hrozek / Alexander Bokovoy, Red Hat Open Source}
\begin{itemize}
\item FreeIPA is an integrated security information management solution
\item FreeIPA integrated with Microsoft Active Directory "out-of-the box"
\item FreeIPA heavy rely on working DNS during install
\item FreeIPA identification server should be heavy secured - its your Security SPOF
\item Kerberos deployment is a fully management Kerberos realm
\item allow AD user to connect to FreeIPA services, todo: allow FreeIPA users to interactively log in into AD machines
\item missing now: CLDAP plugin to FreeIPA to respond on AD discovery quries, KDC backend to generate MS PAC
\item FreeIPA provides smbd, not for sharing (you still can), but for LSA (Local Security Authority)
\item FreeIPA provides nice Web UI configuration/management tool
\end{itemize}

\subsection{New Challenges for Linux Network Support - Marcel Holtmann}
\begin{itemize}
\item Enteprise wants better Ethernet, People wants better wireless
\item on mobile side, we have small tools for too many tasks
\item linux now good at ethernet, wifi personal/enterprise, GSM..LTE, DHCP, IPv4, IPv6, DNS, NTP
\item including tethering, firewall setips, network switching, WISPr
\item New Challenges: Am I actually online? (locally connected vs connected to Internet)
\item ConnMan do "online checking" on global level
\item Every application has different connectivity needs
\item Waking all apps at the same time (when you becomes online) is a really bad idea
\item Even more: mobile devices want different policies for each application
\item Currently Linux haven't grained usage statistics (per app, per user, per interface).
\item It will be nice, to be able say application to use internet, only on wifi connection, and ban mobile just for one app
\item system should not only sync time, but also timezone
\item Marcel Holtmann, Intel pic.twitter.com/5tYwxonT
\item sometimes even per application routing tables are required
\item what about Exclusive usage of VPN? one more thing, which will be nice to have
\item Network connections are global, VPN, Bluetooth tethering - are global
\item netfilter subsystem is too static, its hard to implement firewall policies depending on user/place/app
\item Wifi offload - its hard to automate cases with switching connections to public network with security
\item Bluetooth support pairing via NFC, Wifi support is coming
\item Sensors Network: 802.15.4 and 6LoWPAN is a reality
\item systemd and cgroups will be mandatory for a good mobile networking experience
\end{itemize}

\subsection{Linux Kernel Report - Jon Corbet, Editor at http://LWN.net}
\begin{itemize}
\item Jon Corbet, Editor, http://LWN.net  pic.twitter.com/yBC3vYZO
\item Linux Kernel: Volunteer participation going down, enterprise participation up
\item Companies are very good for hiring kernel hackers...
\item Very positive trend that embedded companies are contributing back to the Linux kernel ecosystem
\item High level of kernel contribution from mobile and embedded players
\item All what is happened in a world with network - happened in Linux
\item Linus main role - is showing attention
\item big.LITTLE - few slow, but power efficient and fast, but big power consumer - cpus on one SoC
\item ext4 still the workhorse filesystem and getting new features
\item btrfs continued stabilization plus send/receive, raid5/6 still waiting
\item F2FS - Flash-friendly filesystem (high performance on NAND flash)
\item CPU scheduler: NUMA scheduling, Power-aware scheduling
\item Kernel problems tend to be: hardware/workload dependent => hard to find automatically => That is what users are for
\item kernel regressions are most important bugs
\item I'm not worried about... The health of Linux as a whole (c) Jon
\end{itemize}

\subsection{Video4Linux2: Path to a Standardized Video Codec API - Kamil Debski, Samsung}
\begin{itemize}
\item Kamil Debski, Samsung pic.twitter.com/cDB1uNHV
\item Linux SoC codec libraries OpenMAX, VA API by Intel, VDPAU by Nvidia
\item V4L2 supports TV tuners, webcams, video capture, output devices
\item memory-to-memory (m2m) devices presents - enables support for video filters, codecs as filters
\item V4L2 Codec API: m2m, multiplanar API, Videobuf2; extensions: h264, h264, mpeg4
\item How to get started with a video coder driver: exisiting codecs in V4L2, flick through other m2m devices
\item successes: API merged in mainline kernel, more hardware video uses V4L2, ARM Chromebook uses V4L2 codec API
\end{itemize}

\subsection{Status of Linux Tracing - Elena Zannoni, Oracle}
\begin{itemize}
\item The Tree of Tracing: ftrace, SystemTap, LTTng, perf, GDB, DTrace
\item Uprobes: Dynamic Userspace Tracing
\item Elena Zannoni, Oracle pic.twitter.com/sXFGx63x
\item Uprobes described as inode, offset in file, list of associated actions, arch specific info
\item Uprobes status: x86, x86\_64(3,5/3.6), powerpc(3.7), ARM(wip)
\item perf/ftrace/systemtap support for uprobes presents
\item Ftrace have many plugins: Function*, Wakeup, Irqoff, Nop...
\item /sys/kernel/debug/tracing/kprobes\_events and /sys/kernel/debug/tracing/uprobes\_events - for CLI control
\item Perf list, annotate, lock, sched, kvm
\item Ftrace-cmd - user space tool, many options, very flexible: record, report, start, stop, extract, list...
\item LTTng - included in embedded Linux distributions
\item SystemTap (Red Hat, IBM, others) - can use debuginfo, dynamic probes and tracepoints, has well defines rich scripting language...
\item DTrace on Linux: solaris tool, porting with goal to support DTrace scripts for Solaris
\item DTrace integrated with Oracle Unbreakable Enterprise Kernel
\item DTrace provider, syscall prvider, SDT, Profile provider, Proc provider, x86\_64 only, Kernel changes - GPL, Kernel modules (CDDL)
\item Tracing: General Open Issues: KABI, Scalability, Code Injection, Embedded community
\end{itemize}

\subsection{DRM/KMS, FB and V4L2: How to Select a Graphics and Video API - Laurent Pinchart, Ideas on Board}
\begin{itemize}
\item Origins: fbdev, Linux 1,3,94 in 1996, Blanking in 2000, 4CC Formats in 2012
\item Origins: DRM Linux 2.2.16 in 2000, GEM in 2008, KSM/TTM in 2009, Dumb Buffers on 2011, Planes, DMABUF in 2012
\item V4L2 in Linux 2.4.0 in 2002, V4L2 subdev in 2008, Media contoller videobuf2 in 2011, DMABUF(?) in 2013
\item there is no any reason to write fbdev drivers nowadays, DRM have full superset of features now
\end{itemize}

\subsection{Optimizing File System Performance When Memory is Tight - Theodore Ts'o, Google}
\begin{itemize}
\item ext4 new features: Punch system call; Metadata checksumming
\item ext4 - "Modern" file system that still reasonable simple
\item ext4 - Incremental devlopment instead of "rip and replace"
\item ext4 well understood performance metrics
\item ext4 disadvantages: fixed inode table, bitmap allocations, 32bit inode numbers
\item ext4 RAID support is extremely weak, lack of sexy new features
\item Ext4 - Default File System for Desktop/Server (distros can change it in near feature)
\item Android devices (Hoeycomb / Ice Cream Sandwich) - use ext4
\item ext4 "is common" in cloud storage servers
\item The economics of cloud computing - run many tasks on single server - memory becomes a bottleneck
\item Restricted memory means less caching available
\item CPU can be a problem too, for PCIe attached flash / trasnscoding video
\item ZFS Open Solaris list suggest 8Gb RAM jusr *for filesystem*
\item Avoiding latency makes the users happy, a few slow requests slow requests behind them
\item Google tests FS with focus on latency, not on avg performance
\item Don't pay for features you don't need - no journal mode for ext4 in Google
\item A lot of metadata caching have done in ext4
\item ext4 sparce files improvements are coming; inline data; raid stripe awareness; atomic msync
\item Conclusion: Remember to optimize the entire storage stack: thin-provisioned snapshots, dm-cache/bcache; optimize userspace
\end{itemize}

\subsection{Rygel: Open Source DLNA, ready for Customer Products? - Jens Georg, Openismus GmbH}
\begin{itemize}
\item .. smth good which comes from MeeGO Disaster
\item UPnP(-AV) Technology \& Devices and Services
\item DLNA - vendor independent. set of media formats, on top UPnP-AV
\item Rygel - UPnP Rendereer and AV-server
\item DLNA picky on calling Rygel DLNA-compatible. Funny guys, my certified Hifi equipment fails quite often 
\item 9h Pieter Hollants @pfhllnts
\item Rygel aims to: Be small, works out-of-the-box
\item Nokia's N9 Smartphone - certified
\item Rygel+DLNA have upload feature
\item Rygel have Media server and renderer plugins
\item SDK: librygel-server/renderer/render-gst
\end{itemize}

\section{Resource Management}
\subsection{Resource Isolation: The Failure of Operating Systems \& How We Can Fix It - Glauber Costa, Parallels}
\begin{itemize}
\item Glauber Costa, Lead Software Developer, Parallels pic.twitter.com/fJtUqjmw
\item history books tell us that back in the day, a computer ran a single program (c) Glauber
\item VM may page out or give less CPU for your app, that you need
\item "forking" app (in sum) gets more CPU than single-thread app
\item Easy DOS: \$ whole true; do mkdir x; cd x; done Will consume all memory before any disk quota can kick in
\item hupervizors(KVM as example) fixes this resource isolation: 1. fair memory 2. N:1 CPU mapping
\item But typical user wants run web, mail servers, databases - to many Virtual Machines for fair resource isolation
\item but sometimes users wants run different versions of Linux for some apps. Solution - containers
\item containers, network namespaces: 40 processes connected to port 80? No problem
\item unique IP, raw devices for each group, per app packet filtering
\item mount namespaces, user namespaces (more that one "root" user in the system)
\item cgroups allows you logical grouping of processes. systemd is heavy uses it
\item CRIU - Checkpoint/Restore complex processes mostly with userspace code, with the aid of some new infrastructure in the kernel
\item cpu/memory controller can do not only fair resource isolation, but also isolation
\item containers, today: openvz - stable, secure and mature Open Source, alternative - LXC, but its still not production ready
\item Work Status: all works: cgroups, CPU, network; need improvements: user/mount/pid/blockIO/CR namespaces
\item To be merged: fork bomb prevention, group-aware kernel memory, filesystem; specialized loop device
\item containers tools: libvirt, vzctl, systemd, etc
\item Note vzctl >= 4.0 works with both OpenVZ and upstream kernels RT containers tools: libvirt, vzctl, systemd, etc
\item kmem controller is already in Andrew Morton's -mm tree
\end{itemize}

\subsection{Systemd: The First Two Years - Lennart Poettering, Red Hat}
\begin{itemize}
\item Lennart Poettering, Red Hat,  pic.twitter.com/9BtOeFQ7
\item systemd is a default in most distributions, included in others as optional
\item systemd community quite a big 15/374 commiters/contributers
\item systemd can now boot system shell-free
\item systemd can boot system in less than 1 second
\item systemd is a Init System; systemd is a Platform
\item systemd: PID 1 does Unit Control Basic Set of Auxiliary Services do the rest
\item systemd loves Embedded hackers, but Lennart personally won't hack embedded features (I asked Q: systemd busybox status)
\item systemd intergrated with dracut, udev, D-Bus, Plymouth, Gummiboot
\item systemd focus on movile, embedded, desktop, server
\item when people brings patches to systemd to support uclibc, in most cases its better to implement such features in libc
\item obsoltes: ConsoleKit, sysvint, initscripts, pm-utils, inted, acpid, syslog, watchdog, cgrulesd. Later on: cron, anacron, atd
\item systemd control PM (power button, led) only util other program (Gnone/KDE) says: "No I control the buttons"
\item systemd: Bring back Unix:Multi-Seat, Services as a File, Set of componenets that are integrated yet separate
\item The Journal Structured, Indexed, Secure, Reliable, Modern, ...
\end{itemize}

\subsection{Checkpoint and Restore: Are We There Yet? - Pavel Emelyanov, Parallels}
\begin{itemize}
\item C/R helps with Live migration / reboot-less kernel update
\item less popular scenario: start-up boost, working enviroment snapshots, HPC loadbalancing
\item OpenVZ could do live migration since 2005 (implemented 100\% in kernel)
\item OpenVZ(Parallels?) failed to upstream their work - they working on it since 2008 to 2010
\item Next idea - reimplement work in usespace, and improve kernel interfaces
\item CRIU ultimate goal: any app -> dump -> restore
\item Exiting kernel APIs: proc, system calls (about self, about anybody), netlink
\item But sometimes kernel could not answer correctly. Solution is obvious but not easy - just add required interfaces.
\item CRIU - a project by various mad Russians to perform c/r mainly from userspace (c) Linus Torvalds
\item CRIU kernel features: 1. parasite code injection (works similar to trace api, which gdb for example use)
\item CRIU kernel features: 2. kcmp - check which kernel objects are shared between processes
\item CRIU kernel features: 3. sockets dumping via netlink - externdable sockets retrive engline
\item Pavel Emelyanov, Parallels pic.twitter.com/IYqhBNm3
\item CRIU kernel features: 4. TCP repair mode
\item CRIU improved: virtual net devices indexes, proc map\_files, socket peeking offset, more socket options
\item CRIU features: x86\_64, process tree linkage, multi-thread, mmap, terminals, groups, sessions, open files, lxc, ...
\item CRIU tests every API separately, but also have tests some big applications (apache, mysql, nginx, mongodb, ...)
\item CRIU plans: Full OS resources coverage, vanilla kernel support, crtools integration with LXC and OpenVZ, live migration, speedup
\item You can easy find CRIU in internet http://criu.org/Main\_Page 
\item CRIU shoild write a FAQ, and first answer should be about security
\end{itemize}

\subsection{Lightweight Virtualization: LXC Best Practices - Christoph Mitasch, Thomas-Krenn.AG}
\begin{itemize}
\item Christoph Mitasch, Thomas-Krenn AG pic.twitter.com/MCIamHin
\item Hardware Virtualization (Full, Para), Software Virtualization(containers)
\item cgroups - the hearth of all Linux container virtualization techs
\item cgroups /sys have "posix" interface - create cgroup with mkdir, change and attach with echo
\item cgroups could hide "fork" bomb, from the used
\item Kernel Namespaces is a "hands" for containers
\item LXC have nice(easy to understand) tools in userspace
\item LXC networking: no entry, empty, veth(bridge), vlan, macvlan, phys
\item LXC + CRIU support is coming soon
\item LXC is unsecure - "Don't give container root to someone you don't trust"
\item ha-cluster of LXC could be done, with DRDB with LCMC, and activate lxc heartbeat
\item rumors says, that you should use vzctl now with upstream kernel support, instead lxc
\item lxc - test kill -PWR 1 initiates a proper shutdown
\end{itemize}

\subsection{Resource Management With Linux - Bruno Cornec, HP}
\begin{itemize}
\item Bruno Cornec, HP pic.twitter.com/VN0lOKhC
\item DEMO will fail during presentation, because every DEMO fails during presentations
\item resource isolation on 160 threads, 4TB RAM, dozen of disks
\item HP do a lot for more NUMA performance in Linux, first patches merged in 2.6.16
\item cgroups - Dedicated subsystems to manage specific resources: CPU shares, memory, blkio; CPU sets, ns, freezer, c/r
\end{itemize}

\section{Cloud}
\subsection{Mark Shuttleworth, Founder, Canonical pic.twitter.com/MTnjvvHu}
\begin{itemize}
\item Mark speaking about importance of skale out
\item Ubuntu is \#1 Cloud OS
\item "We should focus on operation part of the Cloud" (c) Mark
\item Ubuntu decision to be on every cloud, regardless of business \& philosophy obstacles, is smart. 
\item DevOps hell pic.twitter.com/KZ9vSdZu
\item Juju - is a tool, which helps escape from devops hell
\item juju charm includes puppet chef ... python, perl and much more
\item dev@laptop test@cloud deploy@metal
\item Device in your pocket - is truly personal computer (c) Mark
\item the way he explains it it will introduce us to management hell .. which tool is manageing what hell 
\item Virtual Machine Management Costs becomes less than cost of managing Desktop PC
\item Research.Design.Refine in order to lead, according to Mark Shuttleworth 
\item Ubuntu have thin clients for web and windows apps
\item Final words from Shuttleworth: In very short term, a common version of Linux will be avail on any class of device.
\end{itemize}

\subsection{Mostly Sunny: Why Evernote Runs Their Own Linux Servers Instead of "The Cloud" - Dave Engberg, CTO at Evernote}
\begin{itemize}
\item Evernote have about 400 linux servers
\item Dave Engberg, CTO, Evernote pic.twitter.com/T4oHKOZd
\item Public Cloud is a nice to start business, but later (if success) you should have own servers, or private cloud
\item Evernote use Amazon S3 for static, other staff based on their servers
\item Evernote use SSD on metadata servers
\item Evernote: 10 tb mysql. 10 tb lucene. 380 tb apache httpd. 
\item Evernote use WebDav (apache, mod\_dav) as internal protocol for your data
\item Amazon* is good enough for 90-95\% sartups
\item Evernote do calculations, going to private saves them about 75\% of costs (\$228k -> \$60k for compute + metadata task)
\item "Cleverness is the enemy of stability." -Dave Engberg
\item Evernote's CTO making the case for your own private cloud ... it's cheaper .. wasn't that obvious already 
\item You should use cloud, unless you absolutely sure that you shouldn't (c) Dave, because of business risks
\end{itemize}

\subsection{Linux: At the Forefront - Brian Stevens, CTO and VP Worldwide Engineering at Red Hat}
\begin{itemize}
\item RedHat - 20 Years of Disruption
\item The Red Hat Linux Model: fast upstream development, hardened into enterprise releases
\item Is Open Source is a Business Model? - No, But i is the best *development* model (c) Brian
\item Open Source defines architecture of IT (rails, mongodb, etc)
\item Brian Stevens, CTO and VP Worldwide Engineering, Red Hat pic.twitter.com/sK4hgpOc
\item Linux brings your application to ALL platforms
\item The most successful opensource projects of the past and today created by users and developers, not just companies. - @addvin  EU
\item RHEL certifies 17 clouds (in additional to 3000+ "hardwares")
\item with OpenShift you can develop in Java, Ruby, P*, Node.js. Use git, Eclipse, or .. Web IDE
\item Linux Application Stacks: Security from SELINUX, QoS from CGROUPS, LXC
\item 100,000+ Active applications runs on OpenShift
\item Linux is redefined Storage and Data, think about Gluster, Hadoop, Mongo..
\item 80\% of data doesn't sit in a database. Volume of generated doubles every 3 years - Brian Stevens, Red hat cto 
\item KVM and Linux have a head start: oVswitch, OpenStack and Quantum, Emergning s/w network controllers
\end{itemize}

\subsection{Open Source Cloud Platforms - Marten Mickos, CEO at Eucalyptus Systems}
\begin{itemize}
\item 4 open source sitsers: OpenNebula, OpenStack, CloustStack, Eucalyptus
\item Evolution of cloud types: Public -> Private -> Hybrid -> Mobile
\item If Amazon is Starbucks, Eucalyptus is the espresso machine, says @martenmickos  Europe
\item people goes to private clouds from public clouds and data-center
\item cloud customers look for: BIZ: agility, dependability; TECH: experimentation, participation; FIN: optionality\&control
\item Marten Mickos, CEO, Eucalyptus Systems pic.twitter.com/XchdEaKM
\item open clouds saves 80\% money, if compare with private clouds
\item open cloud features: Innovation, Cross-breeding, Deployments, Contributions, Industry support
\item open cloud: freedom of environment, scale, deployment
\end{itemize}


\subsection{Scaling an Open Source Community: How we Grew the OpenStack Project - Monty Taylor, Manager of Automation and Deployment at HP}
\begin{itemize}
\item OpenStack: Compute, Networkoing, Storage, Dashboard
\item Monty Taylor, Manager of Automation and Deployment, HP, OpenStack pic.twitter.com/cSVIEzSF
\item current clouds is very similar to mainframes
\item AIX, Ultrix, HPUX, Solaris, Irix - they all created just for binding customers to their own hardware - this is why they dead now
\item OpenStack is done by 132 Companies
\item OpenStack - Open Source, Design, Development Community
\item OpenStack "goverment model" is a meritocracy (very popular in OSS btw)
\item OpenStack Code Review/Code Standards/Consistent technology/Automated Testing/Automated Everything/Constant Vigence
\end{itemize}

\subsection{Introduction to oVirt Virtualization Management Platform - Itamar Heim, Red Hat}
\begin{itemize}
\item oVirt - focused on KVM, management tool with Web UI
\item oVirt have classic development model: oVirt => Fedora => RHEL
\item oVirt have small footprint (170Mb Fedora) - to run hypervizor only
\item High Availability, Live Migration, System Scheduler, Power Saver, Maintence Manager, Image Management, Monitoring\&Reporting...
\item oVirt could add hosts just in Web UI, will install all required package themself
\item Itamar Heim, Red Hat pic.twitter.com/RDtGMw21
\item oVirt supports various types of storage: NFC, iSCSI, Gluster, and many more
\item oVirt supports not only VNC, but also Spice
\item oVirt have nice UI for creating pools of VMs
\item oVirt support for http://spice-space.org/  remote desktop protocol. Looks great!
\item oVirt also provides "heavy" permission functionality
\item oVirt heavy integrates glusterfs
\item oVirt also provider user "portal", where user see what (and how) he can use in cluster
\item oVirt also provides a lot of reports about resource usage
\item oVirt have nice RESTful Web Service
\item even more, oVirt have python SDK, very nice CLI with documentation and autocompletions
\item oVirt Engine written on Java, PostgreSQL as database, and can be integrated with AD/LDAP
\item oVirt use libvirt/VDSM on Host/Note, Local/Shared Storage
\item oVirt could run Linux/Windows - have guest agent for them (single sign on plugin for example)
\item oVirt Host Agent - VDSM use KSM - "share" memory between different VMs
\item oVirt High Level Architecture - oVirtWiki http://ow.ly/1Pp7Zg 
\item its easy to write hook, which do smth between VDSM calls libvirt
\end{itemize}

\subsection{The OpenStack Project and the OpenStack Foundation - Eileen Evans, HP}
\begin{itemize}
\item OpenStack started by Rackspace and NASA in July 2010
\item There is a big transition, between use OSS internally and contribute to OSS projects, and start OSS project
\item OpenStack do magor release every 6 month, plans for next release are made at OpenStack summit right after release
\item OpenStack 4 opens: Source, Design, Devlopment, Community
\item OpenStack code is available under Apache 2.0 license, and this is forced for all contributions
\item OpenStack Foundation owns the OpenStack trademark
\item Eileen Evans, HP pic.twitter.com/TN04v1Cc
\end{itemize}

\subsection{Cloud Storage Reloaded: Distributed Filesystems (CephFS and GlusterFS) - Udo Seidel}
\begin{itemize}
\item Shared FS - Normal business for Linux: network NFS/CIFS; shared disk OCFS2, GFS2; parallel/distributed: Ceph/GlusterFS
\item recent attention on distributed storage - cloud hype and big data
\item Operations - important part of the cycle; technical challenge; geeks vs. Enterprise
\item Object based storage - partition, file, unique id
\item Ceph and GlusterFS - sufficient community presence, picked up by Enterprise Linux vendors
\item Ceph and GlusterFS - block storage => storage server (OSD); (POSIX) file system -> Meta data; HA -> replication and distribution
\item Client Part - (POSIX) file system, Storage level
\item Parts of ceph: Cluster monitor, md5 cluster, [ ceph ] , OSD Cluster
\item GlusteFS Storage/Brick server <=> (Meta data I/O, Data I/O). NFS included
\item ceph(now) and glusterfs (soon) could be integrated with QEMU, to store vm images directly
\item challnge one: server vs storage
\item challenge two: infrastructure
\item challenge three: support (service provider/application support)
\end{itemize}

\end{document}
